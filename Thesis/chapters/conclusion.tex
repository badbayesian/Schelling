\chapter{Conclusion}
\label{conclusion}

{\lettrine[lines=3,slope=1pt,nindent=1pt,]{\textcolor{SchoolColor}{A}}{n extended Schelling process} appears to be a somewhat compelling explanation for the underlying dynamics of the Chicago housing market given its ability to reproduce Chicago's level of segregation with the stated preferences of respondents to the General Social Survey. Furthermore, it is surprisingly robust to initial conditions. However, this simulation has room for more complexity motivated by realism.

The current extended Schelling algorithm only has a lower bound level of tolerance. However, it is plausible that there is also an upper bound level of tolerance, as well --- in Beckerian terms, ``taste for diversity.'' Furthermore, tolerance levels are constant across races. In practice, there is significant heterogeneity of racial tolerance for housing segregation, and indeed Blacks seem to have a preference for diversity (or, at least, Whites). Future simulations could easily be built to incorporate these types of heterogeneous preferences. In a similar vein, some cities, like Los Angeles, are not well described by a monocentric price topology. Future simulations could be parameterized to find implied tolerance levels for such cities as well. Although RDI and RII are the most common measures of segregation, future papers could benefit from using more obscure but specialized segregation measures. For example, the Racial Exposure Index could be used to measure racial similarity within neighborhoods.

Similarly, looking at other US cities such as Detroit, Houston, and New York City may better illustrate the underlying racial tensions within cities. Chicago is well-known for its corruption and institutional racism in the 20th century, which potentially enforce segregation more strongly than the heavily migrant cities found in Texas, like Houston. In the same vein, Hispanic and Asian immigrants may radically changed the interaction between races. For example, both Asians and Hispanics tend to either segregate heavily when of low wealth, or integrate heavily when of high wealth. This may be more indicative of the \textit{port of entry} theory of segregation after the \textit{structural racism} began to erode with the removal of the Chinese Exclusion Act in 1943 and recent mass Hispanic immigration.

Most importantly, future simulations can build out a more complex model for agent optimization. Currently, an individual faces no trade-off for investing all of one's income, which is endowed each period with 100\% depreciation, into housing. In reality, individuals have at least partially storable income and face tradeoffs between housing and other forms of consumption, as well as other investment opportunities which affect future income flows. Consumption and investment opportunities are affected by one's housing location, but this type of dependency may not add any clarity worth the loss of tractability. R\'emi Lemoy and Charles Raux et Pablo Jensen present a model that could be used for such future simulations\cite{Lemoy10}.  

As this paper shows, however, even honest, benign racial preferences can result in severe segregation exacerbated by wealth inequality. Furthermore, this can occur in the absence of any external pressure from racially motivated policy, past or present. There seems to need to be a preference for diversity if neighborhoods are to desegregate. From a public policy perspective, an ``affirmative action'' policy for neighborhoods is infeasible, but a cultural preference for diversity might not be. Furthermore, governments could institute incentive programs, if not directly for diversity, for factors that correlate with diversity, such as funding for cultural programs but not for close substitutes. This paper confirms Schellings' intuition that the problem of segregation is more difficult than originally conceived. To desegregate, it is not enough solely to \textit{tolerate} other races.