\begin{savequote}[75mm]

\qauthor{}
\end{savequote}

\chapter{Introduction}
\label{introduction}
{\lettrine[lines=3,slope=1pt,nindent=1pt,]{\textcolor{SchoolColor}{E}}{conomists and social scientists} have a difficult time modeling and predicting the structure and growth of neighborhoods. This presents an important issue in the housing market literature, as paper after paper show that the characteristics of a neighborhood strongly affect the social and economic outcome of its residents \footnote{Racial segregation is correlated with negative outcomes for Blacks\footnote{I use the term "Black" to refer to African-Americans in this paper because most of the surveys in this paper use the term "Black" in their terminology.} in academic performance, criminal behaviors, educational attainment, employment, health outcomes, and, more simply, poverty. A short literature includes: Galster, 1987 \cite{galster87}; Case 1991 \cite{case91}; Borjas 1995 \cite{borjas95}; Orfield and Eaton, 1996 \cite{orfield97}; Shihadeh and Flynn, 1996 \cite{shihadeh96}; Cutler and Glaeser, 1997, 1999 \cite{cutler97,cutler99}; Williams and Collins, 2001 \cite{william01}; Card and Rothstein, 2006 -- 08 \cite{card06,card07,card08}.}. While researchers have studied this topic across a wide range of fields, many models of housing markets suffer from a poor balance between modeling the structural dynamics of neighborhoods, which potentially overfits the data, and omitting variables, such as underlying preferences in communities. This paper will first examine surveys to attempt to uncover the general preferences of different races. With this, this paper will examine the mismatching in survey data and empirical data through a simple housing market and additional Schelling model. Due to constraints, this paper will mostly look at Black-White racial relationships.
% * <ejz@uchicago.edu> 2016-05-11T03:39:48.550Z:
%
% > The paper will first examine surveys to attempt to uncover the general preferences of different races. With this, the paper will examine the mismatching in survey data and empirical data through a simple housing market and additional Schelling model. Due to constraints, this paper will mostly look at Black-White racial relationships.
%
% surveys are background; this does not belong in the summary. frankly I would omit this from the introduction period
%
% ^.

Several surveys in the past decades indicate that Black people are, at least, willing to live in a racially heterogeneous community. In 1982, the General Social Survey asked Blacks: ``If you could find the housing that you would want and like, would you rather live in a neighborhood that is all Black; mostly Black; half Black, half White; or mostly White?’’. On average, 67\% of Blacks chose neighborhoods that were either half Black and half White or mostly White \cite{davis92}. Similarly in the 1990s, the Multi-City Study of Urban Inequality (MCSUI) found that 50\% of Black interviewees expressed 50-50 neighborhoods as most attractive, and 99\% of them indicated a willingness to move into such neighborhoods\cite{krysan02}. Pew's 2009 reports suggest that Black people's opinion about White people has not changed much since 1990. Put another way, Black people are more likely to want to live in a heterogeneous neighborhood, and arguably even prefer white neighborhoods. However, these results could be more of an indicator that many Blacks associate White people with wealthier neighborhoods and more amenities rather than an amicability towards White people. \textit{Boondocks}, the popular comic strip and television series,  best characterized this view in their television debut with the line, ``How many times have I told you; you bet' not even dream of tellin' white folk the truth? You understand me? Shoot... makin' white people riot. You better learn how to lie like me. I'm gonna find me a white man and lie to him right now. [sic]"\footnote{One of the tritagonists, Huey is dreaming of showing White people how much their society is whitewashed the lines like "Jesus was Black", etc. He causes all the White people in the vicinity to start to brawl in disbelief of what Huey is telling them.  However, his grandfather, Robert "Granddad" Freeman wakes him up abruptly and delivers the quote above.}\cite{mcgruder05}. 

On the other side, White people have generally been less enthusiastic about integration than Black people, although White people have recently become more accepting of integration. In the previous MCSUI survey in 1990s, 60\% of White people felt comfortable with neighborhoods with one-third Blacks residences and 45\% of whites were willing to move into such neighborhoods \cite{charles03}. The General Social Survey similarly shows that White people are less eager about 50-50 type neighborhoods than Blacks, but have warmed to the idea over time \cite{schuman97}. Overall, White people are increasingly agreeing with the principle that Blacks should be able to live wherever they can afford, although a substantial minority still have their reservations\cite{schuman97,bobo01}. This echoes similarly to the logic of ``Not in my backyard'' where White people say they want Blacks to succeed --- just not near them. It is worth noting that evidence from Detroit seems to indicate that Whites are more willing to remain in their neighborhoods as Blacks enter than they used to be \cite{farley94}. However, when studies found a clear White preference for segregation, the housing market opportunities of Black and Hispanic households were substantially reduced \cite{farley94,bobo96,charles00}. It is also possible, however, that Whites are understating their taste for discrimination in these surveys. Similarly, the recent popularization of politically correct language may mask the true opinions and tastes for discrimination with politeness and platitudes.

Regardless, there is a strong discrepancy between survey responses on race preferences and the actual composition of neighborhoods. The actual composition of neighborhoods is driven by more than just traditional drivers of housing choice, like income. Rosenbaum finds that, after controlling for demographic and economic differences, Blacks reside in lower quality housing and neighborhoods than Whites\footnote{For more information on race preferences see Farley et al. (1978, 1994)\cite{farley78,farley94,farley94b}} \cite{rosenbaum99}. 1990 Census data suggests that the segregation between Blacks and White people in many US cities cannot solely be explained by racial income differences\cite{massey93}. The 1989 Housing Discrimination Study (HDS) found that in both rental and owner-occupied housing markets, Blacks faced significant levels of discrimination\cite{yinger95}. The HDS also conducted their study again in 2000 and found that disparate treatment discrimination in rental and owner-occupied housing markets has been reduced by an order of magnitude over the last decade. However, they find key exceptions to their result in discrimination against Hispanics in access to rental housing, racial steering of Blacks, and less assistance to Hispanics in obtaining mortgages. Other researchers have also found evidence of similar discriminatory behaviors within US housing markets and only modest steps towards integration in the 2000 census\cite{glaeser00,ross05a,turner05}. What is clear is that there is some additional factor driving the market. It is possible that there are other preferences correlated with race, or that there is some hidden financial reason that explains the difference. 

The housing market literature has mixed results on what specific mechanisms cause segregation in housing markets. Cutler, Glaeser and Vigdor (1999) find that segregation has declined since 1970, as Blacks moved into previously all-White areas of cities and suburbs \cite{cutler99}. To some extent, the decline of segregation could be attributed to the Fair Housing Act in 1968, among other legislation, which allowed Blacks to more freely choose their homes. While overall segregation has been reduced, there are also more all-black neighborhoods than before, which drives an increase of the variance of segregation levels of US cities while lowering the mean. Regardless, Cutler et. al. (1999)  find evidence that in 1990 and beyond, the legal barriers which enforced and perpetuated segregation were mostly replaced by a decentralized racism, where whites pay more than blacks to live in the predominantly white neighborhoods \cite{cutler99}. These findings, as well as Bill Rankin's Maps of Chicago, are the primary motivation for this paper.
